\documentclass[11pt]{article}

%%% Additional packages I am using to have more mathematical symbols and commands.
\usepackage{amsmath}
\usepackage{amssymb}
\usepackage[shortlabels]{enumitem}
\usepackage{setspace}
\usepackage[letterpaper, total={6.5in, 9in}]{geometry}

\title{Project Outline: Fast Fourier Transform (FFT)\\ Solving the 1D/2D Poisson Equation with Periodic Boundary Conditions}
\author{Caiwei Zhang, Yitong Wang}
\date{\today}
%%%
\begin{document}
\maketitle
%%%
\doublespacing
\setlength{\parindent}{0pt}

\section*{Goals}
\begin{itemize}
    \item Derive the mathematical steps to convert the Poisson equation into the frequency domain.
    \item Implement a numerical solution using FFT in Julia for both the 1D and 2D cases.
    \item Analyze the computational efficiency and accuracy of using FFT for solving the Poisson equation compared to other methods like Finite Difference Method.
\end{itemize}

\section*{Mathematical Outline}

\subsection{Fourier Transform and FFT}

The Discrete Fourier Transform (DFT) of a sequence \( \{x_n\} \) of \( N \) complex numbers is defined by
\(
Y(k) = \sum_{k=0}^{N-1} x_k w_n^{mk}, \ where \ w_n^{k} = e^{-i \frac{2\pi k}{N}}
\), 
for \( m = 0, 1, \dots, N-1 \), which represents the frequency index. 
The inverse DFT is given by
\(
x_k = \frac{1}{N} \sum_{m=0}^{N-1} Y(m) w_n^{-mk},  k = 0,1, ..., N-1
\).
The FFT algorithm reduces the computational complexity of calculating the DFT from \( O(N^2) \) to \( O(N \log N) \)
by using a divide and conquer approach to break down the DFT computation.

\subsection{Solving the Poisson Equation in Frequency Space}
For the 1D Poisson equation with periodic boundary conditions:
\(
\frac{d^2 u}{dx^2} = -f(x),
\)
we will start by expressing \( u(x) \) and \( f(x) \) in terms of their Fourier series:
\(
u(x) = \sum_{k} \hat{u}(k) e^{i k x}, \quad f(x) = \sum_{k} \hat{f}(k) e^{i k x},
\)
where \( \hat{u}(k) \) and \( \hat{f}(k) \) are the Fourier coefficients of \( u(x) \) and \( f(x) \), respectively.

Taking the Fourier transform of both sides of the Poisson equation, we get
\(
-(k^2) \hat{u}(k) = \hat{f}(k).
\)
In the case where \( k = 0 \), we handle the zero-frequency component separately to ensure the solution's periodicity.

For the 2D Poisson equation:
\(
\frac{\partial^2 u}{\partial x^2} + \frac{\partial^2 u}{\partial y^2} = -f(x, y),
\)
the Fourier transform approach yields
\(
-(k_x^2 + k_y^2) \hat{u}(k_x, k_y) = \hat{f}(k_x, k_y),
\)
and thus,
\(
\hat{u}(k_x, k_y) = -\frac{\hat{f}(k_x, k_y)}{k_x^2 + k_y^2} \quad \text{for} \quad (k_x, k_y) \neq (0, 0).
\)

\section*{References}
\begin{itemize}
    \item Heath, M. T. \textit{Scientific Computing: An Introductory Survey}, Chapter 12 on FFT and its applications.
    \item Julia FFT Documentation: References on using FFT for numerical simulations.
    \item Bretherton, C. S. (n.d.). \textit{Fourier Series and the 2D Poisson Equation}. 
    \item Fitzpatrick, R. (n.d.). \textit{Solution of Poisson's Equation}. 
    \item Roberts, S. J. (n.d.). \textit{Signal Processing and the Fourier Transform}. 
\end{itemize}


\section*{Planned Submission}
We plan to submit a written report that includes a mathematical background of FFT, derivation and implementation details for solving the Poisson equation, and results comparing computational efficiency. 
Visualizations of the solutions for both 1D and 2D cases will be included to demonstrate the effectiveness of this method.

\end{document}
